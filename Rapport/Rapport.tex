\documentclass[a4paper]{report}   

\usepackage[utf8]{inputenc}
\usepackage[francais]{babel}

\usepackage{multirow}

\usepackage{graphicx}
\usepackage{amsmath}
\usepackage{hyperref}

\usepackage{cite}


\usepackage[french]{algorithm2e}

\usepackage{enumerate}

\title{Détecter des données malicieuse à l'aide d'un Réseaux de Neurones }
\author{Liang Wang, Guodong Sun}
\date{\today}

\renewcommand{\arraystretch}{1.2}

\usepackage{etoolbox}
\apptocmd{\thebibliography}{\raggedright}{}{}

\begin{document}

\begin{titlepage}
	\centering
	\vspace{1cm}
\begin{figure}
	\centering
	\includegraphics[scale=0.2]{img/logo_TPT.png}
\end{figure}
	\vspace{1cm}
	{\scshape\LARGE Télécom ParisTech \par}
	\vspace{1cm}
	{\scshape\Large Projet de filière SR2I \par}
	\vspace{1.5cm}
	{\huge\bfseries Détecter des données malicieuse à l'aide d'un Réseaux de Neurones\par}
	\vspace{2cm}
	{\Large\itshape Liang Wang, Guodong Sun \par}
	\vfill
	encadrés par\par
	Jean-Philippe \textsc{Monteuuis}
	\vfill

% Bottom of the page
	{\large \today\par}
\end{titlepage}
	
\begin{abstract}
	
\end{abstract}

\chapter{Introduction}	
	
\section{La voiture connectée en 5G}
Aujourd'hui, la technologie de communication et information a été bien développée et intégrée aux quotidiens. 
Un exemple est la application aux circulations civil. Sans doute, les trafics et les sécurité des conducteurs vont été bien améliorés
Les voitures connectées sont une solution de transport de l'avenir pour les utilisateurs et le gouvernement.  
Le but de la Commission européenne est de connecter tous les véhicules neuf à l'Internet à l'horizon 2022.


Technologiquement, la 5G permettra aux véhicules d'échanger des informations via la connectivité entre véhicules et infrastructure entourant.  
Grâce à la connectivité, les conducteurs pourront mieux savoir les informations en temps réel sur la circulation, les accidents et les conditions environnementales affectant à conduire. 

Mais en réalité, il y a des situation où les données sont sabotées. 
Ou plus gravement, les données sont peut-être modifiées par les hackers.
En ces cas, les résultats sont fatals pour les conducteurs. 
Pour éviter les situations, il faut garantir que les données reçues soient correctes.
Par conséquent, il est nécessaire que on détecte les données malicieuse précisément et effectivement. 
Dans cet article, on recherche les données malicieuse et présente une solution. 

\section{Les réseaux de neurones}
Les réseaux de neurones sont un système de l'algorithme de l'apprentissage par l'expérience.  

\section{Structure de l'article}
Bref, le article utilise les réseaux de neurones pour classifier les données normales et malicieuses.
Tout d'abord, on analyse les caractères dans une paquets de transmission.
Ensuite, on applique les models RNN et RNN + LSTM sur les données. 
Après l'apprentissage et la validation. On compare les résultats des models avec l'autre méthode comme SVM. 
à la fin, on conclure le projet.  

\chapter{Analyse du Paquet Typique}

\chapter{Simulation}




\bibliography{bibliographie}
\bibliographystyle{unsrt}

\listoffigures
\begingroup
\let\clearpage\relax
\listoftables
\endgroup

\end{document}
